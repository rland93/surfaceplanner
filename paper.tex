\documentclass[letterpaper, 10 pt, conference]{ieeeconf}
\usepackage{graphics} 
\usepackage{amsmath} 
\usepackage{amssymb}
\usepackage{optidef}
\usepackage{derivative}
\title{Improving RRT* for a UAV with Dynamic Constraints in 3 Dimensions Using an Optimal Embedded Surface}
\author{Mike L. Sutherland, David A. Copp}

\begin{document}

\maketitle

\subsection{Definitions and Assumptions}
We assume a plan is computed over a 3-dimensional configuration space in $\mathbb{R}^3$. The first two dimensions of the configuration space are represented by a discrete grid, $\bar{X}$, which is a regular grid with each cell having length and width $d$. So, the first two dimensions of the configuration space are fixed in place as grid lattice points. The last dimension of the c-space in the $z$-direction is a real number which corresponds to the altitude of the grid point.

Therefore, the complete configuration space is given as an approximate surface, with each discrete grid point in $X$ corresponding to some real altitude $z$ -- forming the complete configuration space $\mathcal{X}$. We denote a point belonging to $X$ as $x$ and a point belonging to $\mathcal{X}$ as $\bar{x}$. We denote a point belonging to $\mathcal{X}_3$ as $z$.

To query the altitude of some off-grid point, it is assumed that a reasonable interpolation (nearest, linear, polynomial, etc.) can be performed to obtain an intermediate value of $z$.

\subsection{Algorithm}

At a high level, the steps of the algorithm are as follows:

\begin{itemize}
    \item Solve the convex problem
    \item Use the solution to the convex problem to inform the cost function of the RRT*
    \item Solve a 2-D RRT* problem with an altered cost function, $c(\bar{x})$, to obtain an optimal course over the sheet.\footnote{
My intuition says that the RRT cost function is isometric with the distance metric of the configuration space. This *seems* like a prerequisite for the RRT to approach optimality. And, that changes to the RRT cost function that preserve isometry between the two metrics are changes which preserve the property of the RRT which results in the RRT approaching optimality.

Like, that isometry between a cost functions $c$ and $c_\text{eu}$, is a *necessary* and *sufficient* condition for the RRT to find an optimal path through some type of euclidean C-space...

I've tried looking at papers that would explain this or proofs, but everything is so applied -- most use variants of the euclidean cost (if they change it at all), and they do not discuss the properties of the cost function that they use.

I think there is something very interesting and important about the cost function of the RRT that relates to its ability to solve this non-convex problem almost - optimally. Most papers are focused on getting the solution to converge faster on the euclidean metric, but hardly any are looking at the properties of the cost function! I think it could be very exciting to explore that more formally. I think it's crucial for actually explaining how solving the first convex problem, then using that solution inside of the RRT cost, makes the RRT converge to a better solution -- one which is informed by the information obtained from the convex problem.}
\end{itemize}

Each $x \in X$ has some terrain below it, with some height $z_\text{obs}$. Our goal is to compute a real-valued plan elevation, $z$, for all $x \in X$, bearing in mind $z_\text{obs}$, our vehicle dynamics as they relate to the $z$-dimension, and any additional heuristics.

This convex problem can be written:

\begin{mini!}
{J}{J = z}
{\label{eq:Example1}}
{}
\addConstraint{ z }{ \geq z_\text{min} }
\addConstraint{ z - z_\text{gap} }{ \geq z_\text{obs} }
\addConstraint{ \pdv{z}{x} }{ \leq v_\text{max} }
\addConstraint{ \pdv[2]{z}{x} }{ \leq a_\text{max} }
\end{mini!}

Where $z$ is the vehicle height and $z_\text{obs}$ is the obstacle height. $v_\text{max}$ is the maximum climb rate of the vehicle, $a_\text{max}$ is the maximum vertical acceleration of the vehicle, and $z_\text{gap}$ is the minimum gap between the vehicle height and the height of an obstacle.

Solving the convex problem gives us an optimal cost for each point in the configuration space. We can think of this cost as an ``elevation.'' Indeed, in the UAV inspection task, we are trying to fly as low as possible. Then, computing a plan between $x_1$ and $x_2 \in X$ and projecting onto $\mathcal{X}$ will produce a valid plan which will direct the UAV to a safe and optimal height, even under the vehicle's vertical climb and acceleration constraints.


However, we can also improve our RRT by altering its cost function with an additional heuristic informed by this new elevation data. We use the RRT* algorithm, but replacing the standard euclidean cost function

\begin{equation}
c_\text{eu}(\bar{x}_1, \bar{x}_2) = |(x_2 - x_1)|
\end{equation}


With a new cost function. Let $l_{12}$ be a collection of points in $X$, each of which is some distance from a straight line from $x_1$ to $x_2$,

\begin{equation}
l_{12} = \{x \in X | \min(|x - l_{12}|) \leq u \},
\end{equation}

where $u \geq d$. Algorithms such as Bresenham's Straight Line algorithm can be used to compute such collections quickly. The z-values of these points (in $\mathcal{X}$-space) are their ``elevations'' on the sheet. Call this collection of z-values $\mathcal{Z^\text{line}}$. We then plan in $X$ using the cost function
\begin{equation}
c_\text{sheet}(x_1, x_2) = |x_2 - x_1| + \gamma_z \cdot |x_2 - x_1| \cdot \sum^k_i({\frac{z_i}{k}})
\end{equation}

\footnote{
Probably, we need to prove that this cost function forms a metric space $(\mathcal{X}, c)$: positivity, symmetry,  triangle inequality. I think I can use some clever properties of each term and the composition thereof to do this succinctly.}




Where $\gamma_z$ is a scaling factor, $|x_2-x_1|$ is the $\mathbb{R}^2$ distance between the points, and $z_i$ is the elevation of the $i$ th point in a collection of $k$ points formed by the line $l_{12}$. Note that when the scaling factor $\gamma_z=0$, $c_\text{sheet}$ reduces to $c_\text{eu}$. This factor represents a cost penalty on the elevation of each point between the points $x_1$ and $x_2$. 

\end{document}
